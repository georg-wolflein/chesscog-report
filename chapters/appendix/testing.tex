% !TeX root = ./testing.tex
\documentclass[../../report.tex]{subfiles}

\begin{document}

\chapter{Testing summary}

\section{Automated testing}
All three repositories contain automated tests.
Instructions for executing these tests are provided in \cref{sec:recap_tests,sec:chesscog_tests,sec:chesscogapp_tests}.
As explained in \cref{chap:implementation}, they are run on every commit by a \gls{ci} pipeline.
\Cref{fig:tests_recap,fig:tests_chesscog,fig:tests_chesscogapp} show the command line output obtained when running each repository's test suite manually.
\begin{figure}
    \verbatiminput{data/tests/recap.txt}
    \caption{Automated tests for the \texttt{recap} package.}
    \label{fig:tests_recap}
\end{figure}%
\begin{figure}
    \verbatiminput{data/tests/chesscog.txt}
    \caption{Automated tests for the \texttt{chesscog} package.}
    \label{fig:tests_chesscog}
\end{figure}%
\begin{figure}
    \begin{subfigure}[b]{\textwidth}
        \verbatiminput{data/tests/chesscog-app.txt}
        \caption{tests for the backend \acs{api} (Python)}
    \end{subfigure}
    \bigskip\par
    \begin{subfigure}[b]{\textwidth}
        \verbatiminput{data/tests/chesscog-app-node.txt}
        \caption{tests for the frontend \acs{gui} (Node)}
    \end{subfigure}
    \caption{Automated tests for the web app.}
    \label{fig:tests_chesscogapp}
\end{figure}%
These outputs show that the submitted versions of each of the packages pass all the tests.
The Python tests were written using the \texttt{pytest} framework and the frontend tests for the web app use \texttt{jest} for Node.
To test the frontend \gls{gui}, the \texttt{jest} tests mock up the backend \gls{api} with fake responses in order to test that the desired output is rendered to the \gls{dom}.

\end{document}