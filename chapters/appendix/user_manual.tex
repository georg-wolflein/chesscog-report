\documentclass[../../main.tex]{subfiles}

\begin{document}

\chapter{User manual: \texttt{recap} package}
\label{chap:user_man_recap}
\section{Implementation}
\label{sec:user_man_recap_implementation}
Another core feature of the configuration system is a custom way of specifying paths.
Instead of using relative or absolute paths\footnote{Both variants are associated with issues: absolute paths may differ between machines, and relative paths break when certain dependencies (such as the dataset) are located somewhere else.} in the configuration files, \texttt{recap} utilises a format similar\footnote{The format of \texttt{recap} paths does not completely follow the requirements of a \gls{uri}, but we will still refer to them as `\texttt{recap} \glspl{uri}'.} to \glspl{uri} (although the use of conventional paths is still possible).
First, the user registers so-called \emph{path handlers} that translate specific \texttt{recap} \glspl{uri} to absolute or relative paths.
Path handlers are registered for a specific \gls{uri} scheme\footnote{The scheme of a \gls{uri} is the part before the colon and double forward slash (\texttt{://}).}, so if we create a path handler for the \texttt{data://} scheme pointing to \texttt{/path/to/dataset}, the \texttt{recap} \gls{uri} \texttt{data://images/train} will be equivalent to \texttt{/path/to/dataset/images/train}.
These \glspl{uri} can be used in the configuration files as well as the code itself; 
\texttt{recap} provides the \texttt{URI} class which is fully compatible with Python's native \texttt{pathlib.Path} interface and lazily translates the \glspl{uri} to paths on the host system.

\todo{}

\section{Documentation}
\label{sec:recap_documentation}
\todo{}

\chapter{User manual: \texttt{chesscog}}
\label{chap:user_man_chesscog}

\section{Documentation}
\label{sec:chesscog_documentation}
\todo{}

\section{Data synthesis}
\label{sec:user_man_chesscog_data_synthesis}
\begin{verbatim}
cd /Applications/Blender.app/Contents/Resources/2.90/python/bin
./python3.7m -m ensurepip
./python3.7m -m pip install --upgrade pip
./python3.7m -m pip install python-chess
\end{verbatim}
\todo{download dataset}

\chapter{User manual: web app}
\label{chap:user_man_chesscogapp}


\end{document}