\documentclass[../report.tex]{subfiles}

\begin{document}

\chapquote{You may learn much more from a game you lose than from a game you win.}{José Capablanca}
\chapter{Conclusion}
\label{chap:conclusion}

People have long been fascinated by the interaction between machines and humans as it relates to chess.
In the late 18\textsuperscript{th} and early 19\textsuperscript{th} centuries, a chess-playing automaton known as the Mechanical Turk was a popular attraction toured across Europe and the Americas, playing strong games against prominent opponents such as Napoleon Bonaparte and Benjamin Franklin.
Later, it was discovered that the Mechanical Turk was in fact a hoax that needed to be operated by skilled chess players \cite{standage2003}.
It was not until 1996 when IBM's Deep Blue famously defeated World Chess Champion Garry Kasparov that a chess algorithm was considered better than human players.
Nowadays, amateur and professional chess players commonly use chess engines to analyse their positions and improve their game.

Motivated by the problem described in the introduction -- the cumbersome process of transferring a chess position from the board to the computer, this report presents an end-to-end chess recognition system that outperforms all existing approaches.
It correctly classifies 99.72\% of the chessboard squares in the test set (see \cref{ft:per_piece}), thus reducing the error rate of the current state of the art\footnote{The current state of the art has achieves a per-square accuracy of 93.45\% \cite{mehta2020} which corresponds to an error rate of 6.55\%, as compared 0.28\% in our system} by a factor of 23.
In other words, it correctly predicts 93.86\% of the chess positions without any mistakes, and when permitting a maximum of one mistake per board, its accuracy lies at 99.71\%.
Furthermore, a means of one-shot transfer learning for adapting the system to new chess sets is developed.
Based on only two photos of a new chess set, it is able to correctly classify 88.89\% of the positions, and the remaining predictions only had one mistake each.
On a per-square basis, that fine-tuned algorithm reaches an accuracy of 99.83\%, even surpassing the accuracy of the current state of the art system mentioned above which was trained on a lot more than two images.
The functionality of the developed system is demonstrated in a web app, and all code used to run experiments is available so that the results can be reproduced independently.

\section{Future work}
With lockdowns imposed due to the ongoing global COVID-19 pandemic, online chess websites and chess channels witness a surge in popularity \cite{dottle2020}.
This development is significantly amplified by the release of the Netflix miniseries \textit{The Queen's Gambit} which became Netflix's ``biggest scripted limited series ever'' \cite{friedlander2020} just four weeks after its arrival in October 2020 and led to a fivefold increase in daily registrations on the online chess website Chess.com \cite{dottle2020}.
However, it is not only online chess that is gaining traction; the auction website eBay saw a 273\% increase in searches for the term ``chess sets'' in the 10 days following the show's intial release \cite{young2020}, indicating a renewed interest in over-the-board games.
Consequently, the outcome of this project becomes even more relevant because the target audience of potential apps employing vision-based chess inference has grown significantly.

A starting point for such an app could be the web app developed for demonstrational purposes as part of this project (\url{https://www.chesscog.com}), although there is still significant work required until it would be ready to roll out to a large number of users.
First, the infrastructure would need to be upgraded, especially for the backend \gls{api} which performs the chess position inference.
Instead of running on a \gls{cpu}, the \gls{api} server should be equipped with a \gls{gpu} in order to take advantage of the performance gains as per \cref{fig:chesscog_speed}.
Furthermore, to scale the system to a higher number of users, there could be multiple backend servers with a load balancer to distribute requests.

Perhaps a more important issue such an app must solve in order to be useful and successful is how the underlying chess recognition system adapts to the users' own chess sets.
This could be achieved using the one-shot transfer learning approach outlined in \cref{chap:adapting}, but would be very difficult to scale because the training (which requires a few minutes on the \gls{gpu}) would need to be performed and saved for each new user.
An alternative and interesting direction of further research could be to investigate online learning approaches where instead of serving a different chess recognition system for each user, there is one global chess recognition system that continually improves based on the data from all users.
Each time the user uploads an image for inference, the system would produce its prediction and then allow the user to correct the position if it was incorrect.
The uploaded photos along with the corrected labels could then be used as training data to continually improve the \gls{cnn}.
This would alleviate the need to have different models for each user, and would produce a chess recognition system trained on many different chess sets.
However, this may lead to other difficulties such as poor training data because some users might assign incorrect, or a worse performance overall because the chess sets are so diverse.

On a more technical note, another direction of further research is to improve the one-shot learning approach by employing new data augmentation strategies.
One possibility could be to generate an approximation of rotating the pieces around their vertical axis by modelling them as cylinders.
From a cropped input image containing a chess piece, we could find the projection of the input image's pixels onto the cylinder in a procedure similar to \citeauthor{sung2008}'s cylindrical models of human faces \cite{sung2008}.
Then, we could rotate the cylinder and project the points back onto the 2D image plane which could be used as an augmented data sample.

Further work could also examine entirely different approaches to chess recognition. 
For example, it could be investigated whether an object detection model could outperform the occupancy and piece classification \glspl{cnn}.
The problem of object detection (predicting bounding boxes around objects) using \glspl{cnn} has already been extensively studied and even made very accessible via reference implementations in Facebook's \emph{Detectron2} software system for PyTorch \cite{wu2019detectron2}  and the \emph{Object Detection \acs{api}} for TensorFlow \cite{tfodapi2020}, but not yet applied to the problem of chess recognition.
As detailed in \cref{sec:automated_labelling}, we provide bounding box information for the chess pieces in our dataset, so our dataset could be used to train such models.

A new cutting-edge approach could be to use a single neural network in a fully end-to-end fashion for predicting the chess position based solely on the input image (i.e. without an explicit algorithm for localising the chessboard).
Recent advances in applying the \emph{Transformer} architecture \cite{vaswani2017} to computer vision \cite{wu2020,anonymous2020} pave the way for a possible approach where the neural network could use a self-attention mechanism to `attend' to different parts of the input image for predicting the output.
More specifically, the neural network would learn to locate each of the 64 squares in the input image in order to perform the inference.

Finally, it should be noted that the applications of the developed chess recognition system span beyond the scenario that motivated its construction in \cref{chap:introduction}.
The system could be applied to real-time automated move recording at chess tournaments, or recording moves from past games from video footage.
Furthermore, it could be used to design a system for playing chess online, but using a physical board to perform the moves.
In such a system, the live stream from a webcam could be analysed using the chess recognition system in order to determine moves performed on the physical chessboard which can then be broadcast to the opponent.
This would constitute a more engaging means of playing chess against friends physically located elsewhere, a situation especially relevant in light of the ongoing COVID-19 pandemic.

\ifSubfilesClassLoaded{%
\printglossary[type=\acronymtype]%
\printbibliography%
}{}%

\end{document}