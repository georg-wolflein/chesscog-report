\documentclass[../main.tex]{subfiles}

\begin{document}
\chapquote{Life is a kind of chess, in which we have often points to gain, and competitors or adversaries to contend with.}{Benjamin Franklin}
\chapter{Implementation}
\todo{why Python/PyTorch?}
explain why chap brief
\section{Approach}
The implementation of this project took place in three main repositories, all of which were made public on GitHub\footnote{The code of each of these repositories is supplied alongside the submission, but can also be accessed at \texttt{https://github.com/georgw777/\{repo\}} where \texttt{\{repo\}} is to be substituted with the repository's name. The author of this report was the sole contributor to these repositories.}:
\begin{itemize}
    \item \texttt{chesscog}: Python implementation of all core components of the chess recognition system: data synthesis (\cref{chap:data_synthesis}), the recogition pipeline including training and evaluation (\cref{chap:chess_recognition}), and the transfer learning approach to adapt to new chess sets (\cref{chap:adapting});
    \item \texttt{chesscog-app}: a proof-of-concept web app to showcase the chess inference engine; and
    \item \texttt{recap}\footnote{\emph{Recap} is an acronym for REproducible Configurations for Any Project, and is available as a Python package on \acs{pypi} at \url{http://pypi.org/project/recap}.}: a Python package for managing configurations and files associated with machine learning experiments.
\end{itemize}
Originally, \texttt{recap} was part of \texttt{chesscog}; however, later it was refactored into its own repository and published on the \gls{pypi} because other people requested to use this configuration system, too.
Separating the codebases of the core implementation (\texttt{chesscog}) and the web app (\texttt{chesscog-app}) 


Explain the use of:
\begin{itemize}
    \item Repositories
    \item CI/CD pipelines
    \item Testing
\end{itemize}

\begin{figure}
    \makebox[\textwidth][c]{
        \begin{tikzpicture}[
            node distance = 3cm,
            every text node part/.style={align=center},
            repo/.style = {
                circle,
                text width=2.5cm,
                inner xsep=0pt,
                outer sep=0pt,
                draw=black
            },
            action/.style = {
                trapezium,
                trapezium angle=70,
                trapezium stretches=true,
                align=center,
                minimum width=2.5cm,
                minimum height=1cm,
                draw=black
            },
            process/.style = {
                rectangle,
                text width=2.5cm,
                minimum height=1cm,
                text centered,
                draw=black
            },
            arrow/.style = {
                ->,>=stealth
            },
            thickarrow/.style = {
                -{Triangle[width=18pt,length=8pt]}, line width=10pt,lightgray
            }
        ]
            \draw[dashed] (0,0) node[below,right,anchor=north west] {\texttt{chesscog} repository} rectangle (8,-6.5);
            \node[action] (chesscog_commit) at (4,-1.5) {};
            \node at (chesscog_commit) {push\\commit};
            \node[process,below left=1cm and 0cm of chesscog_commit] (chesscog_build) {build};
            \node[process,below right=1cm and 0cm of chesscog_commit] (chesscog_test) {run tests};
            \node[action,below=3cm of chesscog_commit] (chesscog_release) {};
            \node at (chesscog_release) {publish\\release};
            \node[process,below=1cm of chesscog_release] (chesscog_pr) {open \acs{pr} in \texttt{chesscog-app}};
            
            \draw[dashed] (0,-8.5) node[below,right,anchor=north west] {\texttt{chesscog-app} repository} rectangle +(8,-9);
            \node[action,below right=1.5cm and 0cm of chesscog_pr] (chesscogapp_merge) {};
            \node at (chesscogapp_merge) {merge \acs{pr}};
            \node[action,below left=1.5cm and 0cm of chesscog_pr] (chesscogapp_commit) {};
            \node at (chesscogapp_commit) {push\\commit};
            \node[process,below=1cm of chesscogapp_commit] (chesscogapp_build) {build};
            \node[process,below=1cm of chesscogapp_merge] (chesscogapp_test) {run tests};
            \node[draw,diamond,below=6cm of chesscog_pr] (chesscogapp_success) {success and\\on \texttt{master}?};
            \coordinate[above=.75cm of chesscogapp_success] (chesscogapp_success_dummy);
            \node[process,below=1cm of chesscogapp_success] (chesscogapp_deploy) {deploy to\\web server};

            \draw[dashed] (9,0) node[below,right,anchor=north west] {\texttt{recap} repository} rectangle +(8,-6.5);
            \node[action] (recap_commit) at (13,-1.5) {};
            \node at (recap_commit) {push\\commit};
            \node[process,below left=1cm and 0cm of recap_commit] (recap_build) {build};
            \node[process,below right=1cm and 0cm of recap_commit] (recap_test) {run tests};
            \node[action,below=3cm of recap_commit] (recap_release) {};
            \node at (recap_release) {publish\\release};
            \node[process,below=1cm of recap_release] (recap_deploy) {deploy to \acs{pypi}};

            \draw[arrow] (chesscog_commit) -- (chesscog_build);
            \draw[arrow] (chesscog_commit) -- (chesscog_test);
            \draw[arrow] (chesscog_release) -- (chesscog_pr);
            \draw[arrow] (chesscog_pr) -- (chesscogapp_merge);
            \draw[arrow] (chesscogapp_commit) -- (chesscogapp_build);
            \draw[arrow] (chesscogapp_commit) -- (chesscogapp_test);
            \draw[arrow] (chesscogapp_merge) -- (chesscogapp_build);
            \draw[arrow] (chesscogapp_merge) -- (chesscogapp_test);
            \draw (chesscogapp_build) |- (chesscogapp_success_dummy);
            \draw (chesscogapp_test) |- (chesscogapp_success_dummy);
            \draw[arrow] (chesscogapp_success_dummy) -- (chesscogapp_success);
            \draw[arrow] (chesscogapp_success) -- (chesscogapp_deploy) node[pos=0,below right] {yes};
            \draw[arrow] (recap_commit) -- (recap_build);
            \draw[arrow] (recap_commit) -- (recap_test);
            \draw[arrow] (recap_release) -- (recap_deploy);
            
            \draw[thickarrow] (8,-1) -- +(1,0);
            \draw[thickarrow] (1,-8.5) -- +(0,2);
        \end{tikzpicture}
    }
    \caption[Overview of \acs{ci}/\acs{cd} pipelines.]{Overview of \acs{ci}/\acs{cd} pipelines. Trapeziums represent actions initiated by the developer, rectangles correspond to automated processes executed by the pipeline, and the thick arrows represent logical dependency.}
    \label{fig:cicd}
\end{figure}

\begin{figure}
    \makebox[\textwidth][c]{
        \begin{tikzpicture}
            \draw[dashed] (0,0) node[below,right,anchor=north west] {client} rectangle +(8,-11);
            \draw (1,-1) node[below,right,anchor=north west,align=left] {web app displayed in browser} rectangle +(6,-9);
            \draw (2,-2) rectangle +(4,-2) node[midway,align=center] {button to perform\\inference with\\uploaded image};
            
            \begin{scope}[shift={(9,0)}]
                \draw[dashed] (0,0) node[below,right,anchor=north west] {web server (\texttt{chesscog-app})} rectangle +(8,-15.5);
                \draw (1,-1) node[below,right,anchor=north west,align=left] {nginx process} rectangle +(6,-4.5);
                \draw[dotted,thick] (2,-2) rectangle +(4,-1) node[midway] {requests to \texttt{/api/\dots}};
                \draw[dotted,thick] (2,-3.5) rectangle +(4,-1) node[midway] {all other requests};
                \draw (1,-6.5) rectangle +(6,-1) node[midway] {bundled React app (frontend)};
                \draw[<->] (4,-4.5) -- +(0,-2) node[right,pos=.75] {serve static \acs{html} files};
                \draw (1,-8.5) node[below,right,anchor=north west,align=left] {inference \acs{api}} rectangle +(6,-6);
                \draw (2,-9.5) node[below,right,anchor=north west,align=left] {\texttt{chesscog} package} rectangle +(4,-2);
                \draw[dotted,thick] (3,-10.5) rectangle +(3,-1) node[midway] {\texttt{recap} package};
                \draw (2,-12.5) rectangle +(4,-1) node[midway] {Flask server};
                \draw[<->] (2.5,-11.5) -- +(0,-1);
                \draw[<->] (6,-2.5) edge[bend left=60] (6,-13);
                \node[right,fill=white] at (7,-4) {forward};
            \end{scope}

            \draw[->] (6,-3) -- +(5,.75) node[midway,above,sloped,fill=white] {\acs{http} POST \texttt{/api/predict}};
            \draw[->] (11,-2.75) -- (6,-7) node[midway,above,sloped,fill=white] {response};
            \draw[->] (7,-7.75) -- (11,-3.75) node[midway,above,sloped,fill=white] {\acs{http} GET \texttt{/}};
            \draw[->] (11,-4.25) -- (7,-8.25) node[midway,below,sloped,fill=white] {response};

            \begin{scope}[shift={(2,-9)}]
                \foreach \y in {0,1,...,3}{
                    \foreach \x in {0,1,...,3}{
                        \edef\temp{\noexpand\fill[pattern=crosshatch, pattern color=black] (\x,\y) rectangle (.5+\x,.5+\y) rectangle (1+\x,1+\y);}
                        \temp
                    };
                };
                \draw (0,0) rectangle (4,4);
            \end{scope}
        \end{tikzpicture}
    }
    \caption{Schematic overview of the infrastructure pertaining to the app.}
\end{figure}

\section{Data synthesis}
\section{Configuration management}
\gls{pypi}

\section{Board localisation}
\section{Training the classifiers}
\section{Transfer learning}
\section{Web app}


\end{document}