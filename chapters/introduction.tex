\documentclass[../main.tex]{subfiles}

\begin{document}

\chapter{Introduction}
\chapquote{Improving your weaknesses has the potential for the greatest gains.}{Garry Kasparov}

Former World Chess Champion Garry Kasparov remarks the epigraph above in his most recent book \cite{kasparov2018} -- a profound observation that applies even outside of chess.
With regard to chess in particular, Kasparov implies that you must identify your mistakes and weaknesses in order to improve as a player, and to do so, you must analyse your own games. 

Amateur chess players can analyse games they played online without much effort because the moves are recorded automatically.
However, to analyse over-the-board games\footnote{Usually, players will invest more effort in over-the-board games, both in terms of time and deep thinking. These games will also involve a greater psychological aspect as a result of being able to observe the opponent's expressions. As such, analysing these games should be even more interesting and fruitful.}, players must tediously enter the position in the computer piece by piece.
A casual over-the-board game between two friends will often reach an interesting position\footnote{For example, one of the players might have a few moves that look promising, but is also considering a line with a piece sacrifice. If he decides to play it safe, he will likely want to analyse the piece sacrifice on the computer after the game.}. 
After the game, the players will want to analyse that position on a computer, so they take a photo of the position. 
On the computer, they need to drag and drop pieces onto a virtual chessboard until the position matches the one they had on the photograph, and then they must double-check that they did not miss any pieces.

The goal of this project is to develop a system that is able to map a photo of a chess position to a structured format that can be understood by chess engines, such as the widely-used Forsyth–Edwards Notation (FEN) \cite{edwards1994}, in order to automate this laborious task.


\section{Context survey}


Determining the game state of a chess board, also known as \emph{chess recognition}, is a problem in computer vision whereby an algorithm is tasked with recovering the configuration of pieces from an image of a chessboard.
Early work on chess recognition in the 1990s focused on extracting typeset games from printed material \cite{baird1990}. 
In recent years, the problem of parsing two-dimensional chess images has effectively been solved using conventional machine learning techniques \cite{khater2012} and deep learning \cite{sameer2020,roy2020}.
However, recognising chess positions from physical chessboards as opposed to artificial two-dimensional images poses a much more interesting and challenging problem that finds practial application in chess-playing robots, augmented reality, and aiding amateur chess players\footnote{Electronic chess sets are impractical and very costly \cite{wang2013}, thus solutions for chess recognition using just a photo of an unmodified chess board are more compelling for amateur chess players.}.

\vspace{.5cm} % dirty fix because this heading for some reason doesn't consume any space
\paragraph{Chess robots}
Initial research into chess recognition emerged from the development of chess robots that included a camera to detect the human opponent's moves from a top-down overhead perspective. 
The difficulty of distinguishing between chess pieces from a bird's-eye-view due to their similarity is noted in many papers; as a result, chess robots typically implement a three-way classification system that for every square attempts to determine whether it contains a piece, and if so, the piece's colour.
Various approaches have been explored including
  employing manual thresholding \cite{cour2002,urting2003,banerjee2012,chen2016} and clustering \cite{goncalves2005} in different colour spaces, as well as
  differential imaging (classifying based on the per-pixel difference between two images) \cite{khan2014,chen2019}.
Although the \emph{Gambit} robot proposed by \textcite{matuszek2011} does not require a bird's-eye view over the chessboard and uses a depth camera to more reliably detect the occupancy of each square, it employs the three-way classification strategy using a linear \gls{svm} to determine the piece colour. 

\paragraph{Chess move recording}
Several techniques for recording chess moves from video footage have been proposed that follow a similar three-way occupancy and colour classification scheme, both from a top-down perspective \cite{sokic2008,wang2013} as well as from a camera positioned at an acute angle to the board \cite{hack2014}.
However, in any three-way classification approach, the robot or move recorder requires knowledge of the previous board state in addition to its predictions for each square's occupancy and piece colour to deduce the last move. 
While this information is readily available to a chess robot or move recording software, this is not the case for a chess recognition system that should deduce the position from a single still image.
Furthermore, these approaches experience severe shortcomings in terms of their inability to recover once a single move was predicted incorrectly and fail to identify promoted pieces\footnote{Piece promotion occurs when a pawn reaches the last rank, in which case the player must choose to promote to a queen, rook, bishop or knight. Evidently, a vision system that can only detect the piece's colour is unable to detect what it was promoted to.} \cite{cour2002}.

\paragraph{Single-image chess recognition}
A number of techniques have been developed to address the issue of chess recognition from a single image. 
Unlike move recording software or chess robots, it does not suffice to only determine the occupancy and colour of each square, but each piece must be identified.
These techniques must implement a classification algorithm for each piece type (pawn, knight, bishop, queen, and king) of each colour which poses a significantly more difficult problem, attracting research mainly in the last five years.
From a bird's-eye view, the pieces are nearly indestinguishable, so the photo is usually taken at an angle to the board.
\textcite{ding2016} proposes a piece classifier that uses one-versus-rest \glspl{svm} trained on \gls{sift} and \gls{hog} feature descriptors, achieving an accuracy of 85\%. 
\textcite{danner2015} as well as \textcite{xie2018} claim that \gls{sift} and \gls{hog} provide inadequate features for the problem of piece classification due to the similarity in texture between chess pieces, and instead focus on the pieces' outlines.
As such, \textcite{danner2015} use Fourier descriptors calculated for the pieces' contours, but this requires a manually-created database of piece silhouettes.
Furthermore, they modify the board colours to red and green instead of black and white, in order distinguish the pieces from the board more easily\footnote{Similar board modifications have also been proposed as part of chess robots \cite{banerjee2012} and chess move trackers \cite{wang2013}, but any such modification imposes an unreasonable constraint on normal chess games.}.
On the other hand, \textcite{xie2018} perform contour-based template matching with an interesting caveat: the camera angle is calculated based on the perspective transformation of the chessboard, and then depending on the angle, different templates are utilised for matching the chess pieces.
As part of the same work, \citeauthor{xie2018} developed another approach that instead utilised \glspl{cnn}, but found that their original template-matching technique achieved superior results in terms of speed and accuracy in low-resolution images.
However, it is important to note that their \glspl{cnn} were trained on only 40 images per class and deep learning methods tend to excel when trained on larger datasets.

\paragraph{Chessboard detection}
A prerequisite to any chess recognition system is the ability to detect the location of the chessboard and each of the 64 squares. 
Once the four corner points have been established, finding the squares is trivial for pictures captured in bird's-eye view, and only a matter of a simple perspective transformation in the case of other camera positions.
While finding the corner points of a chessboard is frequently used for automatic camera calibration due to the regular nature of the chessboard pattern \cite{delaescalera2010,bennett2014}, techniques designed for this purpose tend to perform poorly when there are pieces on the chessboard that occlude lines or corners.
Some of the aforementioned chess robots \cite{goncalves2005,sokic2008,khan2014} as well as the single-image recognition system proposed by \textcite{danner2015} circumvent this problem entirely by prompting the user to interactively select the four corner points, but ideally a chess recognition system should be able to parse the position on the board without human intervention.
Most approaches for automatic chess grid detection utilise either the Harris corner detector \cite{banerjee2012,hack2014} or a form of line detector based on the Hough transform \cite{tam2008,neufeld2010,danner2015,chen2016,kanchibail2016,xie2018a,chen2019}, although other techniques such as template matching \cite{matuszek2011} and flood fill \cite{wang2013} have been explored.
In general, corner-based algorithms are unable to accurately detect grid corners when they are occluded by pieces, thus line-based detection algorithms appear to be the favoured solution.
Such algorithms often take advantage of the geometric nature of the chessboard which allows to compute a perspective transformation of the grid lines that best matches the detected lines \cite{tam2008,hack2014,xie2018}.
However, lines found in the background of the photo can often cause failure modes.
A recent chess grid detection algorithm that is highly successful even on populated boards is descibed by \citeauthor{xie2018a} in \cite{xie2018a}. 
They apply several clustering algorithms on the lines detected via a Hough transform in order to find the horizontal and vertical grid lines belonging to the chessboard, and use this algorithm as a preprocessing step in their template-matching piece classification technique \cite{xie2018} described above. 

\paragraph{Chess recognition using \glspl{cnn}}
Since \citeauthor{xie2018} pioneered the use of \glspl{cnn} in the domain of chess recognition from monocular images in \citeyear{xie2018}%
\footnote{%
    \textcite{wei2017} developed a chess recognition system using a volumetric CNN one year previously, but this approach requires three-dimensional chessboard data obtained from a depth camera. 
    Their approach achieved a per-class accuracy over 90\% except for the ``king'' class, was trained on \gls{cad} models, and evaluated on real three-dimensional images (point clouds) of a chessboard.
},
a few more techniques have been developed that employ \glspl{cnn} at various stages in the recognition pipeline.
\textcite{czyzewski2020} achieve an accuracy of 95\% on chessboard detection from non-vertical camera angles by designing an iterative algorithm that gereates heatmaps over the input image representing the likelihood of each pixel being part of the chessboard. 
They then employ a \gls{cnn} to refine the corner points that were found using the heatmap, outperforming the results obtained by \textcite{goncalves2005}.
Furthermore, they compare a \gls{cnn}-based piece classification algorithm to the SVM-based solution proposed by \textcite{ding2016} and find no notable improvement, but manage to obtain major improvements by implementing a probabilistic reasoning system that uses the open source Stockfish chess engine \cite{romstad2020} as well as chess statistics \cite{acher2016}.
Although reasoning techniques were already employed for refining the predictions of chess recognition systems before \cite{neufeld2010,danner2015}, \citeauthor{czyzewski2020} demonstrate the potential of combining information obtained from a chess engine with large-scale chess statistics.  
Very recently, \textcite{mehta2020} implemented an augmented reality app using the popular \emph{AlexNet} \gls{cnn} architecture introduced by \textcite{krizhevsky2017}, achieving promising results.
Despite using an overhead camera perspective and not performing any techniques to ensure probable and legal chess positions, \citeauthor{mehta2020} achieve an end-to-end accuracy of 93\% for the entire chessboard detection and piece classification pipeline.

\paragraph{Datasets}
The lack of adequate datasets for chess recognition has been recognised by many \cite{czyzewski2020,ding2016,mehta2020}.
Although \textcite{czyzewski2020} published a dataset of chessboard lattice points that are difficult to predict \cite{czyzewski2018}, large datasets -- especially at the scale required for deep learning -- are not available as of now.
Using synthesised data in the training set is an efficient means of creating sizable datasets while minimising the manual annotation efforts \cite{wei2017,hou,czyzewski2020}.
\citeauthor{czyzewski2020} distort some input images in order to simulate different camera perspectives on the chessboard corners.
However, a more promising method seems to be the use of three-dimensional models.
\textcite{wei2017} synthesise point cloud data for their volumetric \gls{cnn} directly from three-dimensional chess models and \textcite{hou} use renderings of three-dimensional models as input. 
Yet \textcite{wei2017}'s approach works only if the chessboard was captured with a depth camera and \textcite{hou} presents a chessboard recognition system using a simple \gls{ann} that is not convolutional and hence achieves an accuracy of only 72\%.

\section{Objectives}
\label{sec:objectives}
As mentioned in the introduction, the main aim of this project is developing an end-to-end chess recognition system that takes as input an image of a chessboard with pieces on it and outputs the game state (the chess position).
To this end, the \gls{doer} document lists five primary objectives that should define the success of this project:
\begin{enumerate}[label={A\arabic*.},ref={A\arabic*}]
    \item \label{obj:11} Perform a literature review of available methods for parsing chess positions from photos.
    \item \label{obj:12} Develop an algorithm for detecting the corners of the chessboard as well as the squares.
    \item \label{obj:13} Develop an algorithm for recognising the chess pieces.
    \item \label{obj:14} Develop an algorithm that uses the outputs from (2) and (3) in order to compute a probability distribution over each piece in each square.
    \item \label{obj:15} Evaluate the performance of the developed algorithms.
\end{enumerate}

Next, there are three secondary objectives that provide meaningful extensions to the project:
\begin{enumerate}[label={B\arabic*.},ref={B\arabic*}]
    \item \label{obj:21} Create a large labelled dataset of synthesised chessboard images using 3D models.
    \item \label{obj:22} Implement an algorithm that takes as input the raw probability distribution of each piece in each square and outputs a likely \gls{fen} description.
    \item \label{obj:23} Implement a simple web API that performs the inference pipeline for an input image, returning the \gls{fen} description. 
\end{enumerate}

Finally, the \gls{doer} lists one tertiary objective. 
However, during the course of this project, an additional tertiary objective (\ref{obj:32}) was introduced as it became clear that employing a transfer learning approach to adapt to new chess sets would make the system significantly more useful:
\begin{enumerate}[label={C\arabic*.},ref={C\arabic*}]
    \item \label{obj:31} Develop a web app that allows the user to upload an image of the chess board to obtain the \gls{fen} description.
    \item \label{obj:32} Employ transfer learning to demonstrate how the system can adapt to new chess sets.
\end{enumerate}

This project fully achieves all of the objectives listed above, as is evaluated in detail in \cref{sec:achievement_of_objectives}.
An interactive proof-of-concept demonstration of the developed system is available online at \url{https://www.chesscog.com}.
However, this web app should just be regarded as a byproduct, since the main focus of this project lies in the development of the chess recognition algorithm itself.

\Cref{chap:data_synthesis} explains the process used to synthesise a large dataset of chess images using a 3D model of a chess set.
The design of the chess inference system is explained in \cref{chap:chess_recognition}, and \cref{chap:adapting} extends this approach to facilitate adapting to new (previously unseen) chess sets.
\Cref{chap:implementation} discusses the implementation at a high level which is evaluated in \cref{chap:evaluation} not only with regard to the aforementioned objectives, but also compares the system to the existing approaches identified in the context survey.
Finally, \cref{chap:conclusion} concludes the report and provides an outlook for future work.
\section{Ethics}
There are no ethical issues raised by this project, as indicated in the signed ethics form in \cref{app:ethics_form}.

\end{document}