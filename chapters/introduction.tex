\documentclass[../main.tex]{subfiles}

\begin{document}

\chapter{Introduction}

Former World Chess Champion Garry Kasparov remarks that ``improving your weaknesses has the potential for the greatest gains'' \cite{kasparov2018} -- a profound observation that may even apply outside of chess.
With regard to chess in particular, Kasparov implies that you must identify your mistakes and weaknesses in order to improve as a player, and to do so, you must analyse your own games. 

Amateur chess players can analyse games they played online without much effort because the moves are recorded automatically.
However, to analyse over-the-board games\footnote{Usually, players will invest more effort in over-the-board games, both in terms of time and deep thinking. These games will also involve a greater psychological aspect as a result of being able to observe the opponent's expressions. As such, analysing these games should be even more interesting and fruitful.}, players must tediously enter the position in the computer piece by piece.
A casual over-the-board game between two friends will often reach an interesting position\footnote{For example, one of the players might have a few moves that look promising, but is also considering a line with a piece sacrifice. If he decides to play it safe, he will likely want to analyse the piece sacrifice on the computer after the game.}. 
After the game, the players will want to analyse that position on a computer, so they take a photo of the position. 
On the computer, they need to drag and drop pieces onto a virtual chessboard until the position matches the one they had on the photograph, and then they must double-check that they did not miss any pieces.

The goal of this project is to develop a system that is able to map a photo of a chess position to a structured format that can be understood by chess engines, such as the widely-used Forsyth–Edwards Notation (FEN) \cite{edwards1994}, in order to automate this laborious task.


\subfile{introduction/context_survey}
\subfile{introduction/ethics}

\end{document}