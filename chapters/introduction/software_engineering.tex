\section{Software engineering process}
The development of this project is characterised by a very agile approach due to its primarily research-oriented nature.
Regular supervisor meetings were held where progress was discussed, and tasks were set for the next week.
This ensured that changes could be made quickly depending on the outcomes of different approaches that were tested throughout the project.

Even though this project was primarily research-based, good software engineering practices were still followed.
The code developed throughout this project was managed using Git repositories.
The general approach to developing a new algorithm was to first implement it in an interactive Python notebook to allow rapid prototyping, and once development arrived at a working version, it was refactored into one or more Python files.
Keeping the code as Python files instead of interactive notebooks avoids code duplication because many components of the chess recognition system are used both at training and inference time.
Furthermore, this approach was necessary so that the algorithms could be tested using unit tests.
The unit tests for a particular repository are run on every commit to that repository via a \gls{ci} pipeline.
Moreover, since the web demo was developed in parallel to the core chess recognition algorithm, \gls{cd} pipelines were set up for automatically deploying the web app, as well as integrating changes from the core recognition algorithm upon release.
Specific details follow in \cref{chap:implementation}.

An overarching theme that governed the development of the chess recognition system is that of reproducibility.
Experimental results were persisted in data files\footnote{The data files are available in the \path{chesscog/results} folder.} which can be reproduced using the scripts described in \cref{chap:user_man_chesscog} to independently verify the benchmarks and results claimed throught this report.