\documentclass[a4paper,oneside]{book}
\usepackage[utf8]{inputenc}
\usepackage[ngerman]{babel}
\usepackage[table]{xcolor}
\usepackage{quotchap}
\usepackage{breakcites} % don't let citations break out of the text frame
\usepackage[font=small]{caption}
\usepackage{csquotes}
\usepackage{footnote}
\usepackage{fancyhdr}
\usepackage{minted}
\usepackage{verbatim}
\usepackage{amsmath}
\usepackage{amsfonts}
\usepackage{xfrac}
\usepackage{bm}
\usepackage{booktabs}
\usepackage{mathtools}
\usepackage{etoolbox}
\usepackage{makecell}
\usepackage{array}
\usepackage{multirow}
\usepackage{pifont}
\usepackage{xskak}
\usepackage{graphicx}
\usepackage{caption}
\usepackage{subcaption}
\usepackage{tikz}
\usepackage{pgfplots}
\usepackage{pgfplotstable}
\usepackage{pdfpages}
\usepackage{wrapfig}
\usepackage[hidelinks]{hyperref}
\usepackage[inline]{enumitem}
\usepackage[style=ieee,mincitenames=1,maxcitenames=2]{biblatex}
\usepackage[toc,acronym,shortcuts,nopostdot]{glossaries}
\usepackage{glossary-mcols}
\usepackage[capitalise,noabbrev]{cleveref}
\usepackage{algorithm}
\usepackage{algpseudocode}
\usepackage{subfiles}


% BEGIN: left-align quotchap
\makeatletter
\renewcommand*{\sectfont}{\bfseries}% make it bold
\renewcommand\chapter{%
  \if@openright\cleardoublepage\else\clearpage\fi
  \thispagestyle{plain}%
  \global\@topnum\z@
  \null\hfill\@printcites\par%
  \@afterindentfalse
  \secdef\@chapter\@schapter
}
\renewcommand{\@makechapterhead}[1]{%
  \chapterheadstartvskip%
  {\size@chapter{\sectfont\raggedright%
    {\chapnumfont
      \ifnum \c@secnumdepth >\m@ne%
      \if@mainmatter\thechapter%
      \fi\fi
      \par\nobreak}%
    {\raggedright\advance\leftmargin10em\interlinepenalty\@M #1\par}}% NEW
  \nobreak\chapterheadendvskip}}
\makeatother
% END: left-align quotchap


\author{Georg Wölflein}
\title{MSci Dissertation}


\begin{document}
\chapter*{Kurzfassung}
\thispagestyle{empty}
Anhand eines Schachbrettfotos die Stellung der Figuren zu erkennen ist ein Problem in der digitalen Bildverarbeitung, das bisher noch nicht sehr zuverlässig gelöst wurde. 
Mit einem solchen System müssten Amateurschachspieler die Stellung der Figuren für eine Computeranalyse nicht mehr manuell eingeben und können somit ihre Fähigkeiten einfacher verbessern. 
Gegenwärtige Ansätze zur Schachpositionserkennung leiden u.a.\ durch das Fehlen großer Datenmengen und sind nicht darauf ausgelegt, sich an zuvor ungesehene Schachsets anzupassen. 
In diesem Projekt wird ein neuer Datensatz vorgestellt, der Bilder mithilfe eines 3D-Modells synthetisiert und hundertfach mehr Beispieldaten erzeugt. 
Ausgehend von diesem Datensatz wird ein neuartiges Schacherkennungssystem erarbeitet, das traditionelle Computer-Vision-Techniken mit Deep-Learning-Verfahren vereint. 
Das Schachbrett wird mithilfe eines auf RANSAC basierenden Algorithmus lokalisiert, welcher eine projektive Transformation des Bretts auf ein quadratisches Raster vornimmt. 
Anschließend wird unter Verwendung von zwei Convolutional Neural Networks (CNNs) eine binäre Belegungsmaske der Schachfelder im verzerrten Bild festgestellt und schließlich werden die Figuren identifiziert. 
Das beschriebene System erreicht eine Fehlerquote von 0,28\% pro Schachfeld im Testdatensatz, 23 mal besser als der aktuelle Stand der Technik. 
Darüber hinaus wird ein Ansatz basierend auf Transfer Learning entwickelt, wodurch das Inferenzsystem mit nur zwei Fotos der Ausgangsstellung an ein zuvor nicht gesehenes Schachspiel angepasst werden kann. 
Damit lässt sich eine Genauigkeit von 99,83\% pro Feld auf Bildern dieses zuvor unbekannten Schachsets erzielen. 
Die Inferenz dauert mit GPU-Unterstützung weniger als eine halbe Sekunde und ungefähr zwei Sekunden auf der CPU. 
Die Machbarkeit des Systems wird in einer interaktiven Web-App aufgezeigt, die unter \url{https://www.chesscog.com} aufgerufen werden kann.

\end{document}
